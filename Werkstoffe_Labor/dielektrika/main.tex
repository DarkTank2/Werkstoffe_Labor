\documentclass[a4paper,twoside,12pt,DIV=13,BCOR=5mm,numbers=noenddot,cleardoublepage=empty]{scrbook}
\input{format-LU} 
\usepackage{graphicx}
\usepackage{esvect}
\usepackage{bm}
\usepackage{mathtools}
\begin{document}
    \renewcommand{\baselinestretch}{1.25}
    \newcommand{\StudentA}{Philipp Hanser}
    \newcommand{\MatrNrA}{11775264}
    \newcommand{\StudentB}{Florian Strebl}
    \newcommand{\MatrNrB}{11712190}
    \newcommand{\StudentC}{Alexander Seiler}
    \newcommand{\MatrNrC}{11771276}

    \newcommand{\LUDatum}{06.06.2019}
    \newcommand{\LUGruppe}{Gr. 15}
    \newcommand{\LUBetreuer}{Geiginger, Lisa-Marie}  

    \large
    \includepdf[fitpaper=true,
                            picturecommand*={\unitlength1cm 
                            \put(7.3,7.7){\StudentA} \put(14.1,7.7){\MatrNrA}
                \put(7.3,7.0){\StudentB} \put(14.1,7.0){\MatrNrB}
                \put(7.3,6.3){\StudentC} \put(14.1,6.3){\MatrNrC}
                            \put(7.3,5.1){\LUDatum} 
                \put(7.3,4.4){\LUGruppe} 
                \put(7.3,3.7){\LUBetreuer} 
    }]
    {pictures/DeckblattLUDie}


    \setcounter{tocdepth}{3}

    \setcounter{page}{0}
    \renewcommand{\thepage}{\roman{page}}
    \tableofcontents \cleardoublepage

    \setcounter{page}{1}
    \renewcommand{\thepage}{\arabic{page}}
    \setcounter{chapter}{0}

    \chapter{Durchgangs- und Oberfl\"achenwiderstand}
        \section{Grundlagen}
        \subsection{Durchgangswiderstand}
        Grundlegend gibt es zwei Arten der elektrischen Leiter: \\
        \\
        \textbf{Elektronenleiter} \\
        Elektronenleiter werden auch als Leiter erster Klasse bezeichnet.
        Hierbei handelt es sich um Metalle, Legierungen, Halbleiter und
        teilweise einige Organische Verbidnungen. Stromfluss entsteht hierbei durch 
        die Bewegung von Elektronen.
        \\
        \\
        \textbf{Ionenleiter} \\
        Ionenleiter werden auch als Leiter zweiter Klasse bezeichnet. Bei ihnen
        spielt die elektrolytische Leitf\"ahigkeit eine wesentliche Rolle. 
        Stromfluss bei Ionenleiter tritt durch die Bewegung von Ionen auf. Dadurch
        ist ein Stromfluss in Ionenleitern immer mit einem Massefluss verbunden.
        Beispiele f\"ur Ionenleiter sind zum Beispiel ionisierte Gase oder Wasser und Salze.
        \\
        \\
        Der spezifische Durchgangswiderstand gibt das Isolationsverm\"ogen
        eines Dielektrikums an. Er wird in $\Omega$m gemessen. Durch das anlegen
        einer Spannung von 500 Volt, werden die Ladungstr\"ager im Dielektrikum
        an die entgegengesetzt Ladungsoberfl\"ache bewegt. Diese bewegung entspricht
        einem Stromfluss. Da jedoch nur eine begrenzte Anzahl an Ladungstr\"agern
        im Medium verf\"ugbar sind, nimmt der Stromfluss stetig ab und der gemessene Widerstand
        steigt. Deshalb sollte man, um die Proben vergleichen zu k\"onnen, immer nach
        einer festgelegten Zeit messen. 
        \subsection{Oberfl\"achenwiderstand}
        Da die Oberfl\"ache eines Dielektrikums stark mit der umliegenden
        Athmosph\"are wechselwirkt, kann bei Messungen des Oberfl\"achenwiderstandes
        Abweichungen zu theoretischen Werten auftreten. Diese sind vor allem auf Wasser
        in der Luft zur\"uckzuf\"uhren. Deswegen unterscheidet man zwischen
        \textbf{hydrophoben} und \textbf{hydrophilen} Dielektrika. Erstere besitzen
        durch ihre Wasserabweisenden eigenschafften einen hohen relativ konstanten, 
        im Bezug auf die Luftfeuchtigkeit, Oberfl\"achenwiderstand. Letztere Dielektrika
        sind stark von der Luftfeuchtigkeit abh\"angig. Da bei zunehmender Luftfeuchtigkeit
        ein d\"unner Film auf der Oberfl\"ache w\"achst. Dieser ist verantwortlich f\"ur das
        sinken des Oberfl\"achenwiderstandes.
        \section{Aufgabenstellung}
        Im Zuge des Dielektrika-Labors sollen f\"unf verschiedene Dielektrika 
        zum Messen herangezogen werden. Bei jedem dieser Dielektrika sollen 
        jeweils Durchgangswiderstand als auch Oberfl\"achenwiderstand gemessen 
        werden. Die f\"unf Dielektrika sind PVC, Phenolharz, Teflon, Plexiglas und 
        Epoxidharz\\
        Wichtig ist jedoch dass darauf geachtet wird das die Messwerte jeweils zum 
        gleichen Zeitpunkt nach dem Einschalten des Messger\"ates aufgenommen werden.
        In diesem Fall wird jeweils nach einer vollen Minute der Messwert 
        aufgenommen, um die Messwerte der unterschiedlichen Dielektrika vergleichen
        zu k\"onnen.
        \\
        \\
        Die Proben, die gemessen werden, werden jeweils zwischen zwei Elektroden
        mit einem Schutzring herum platziert (siehe Abbildung Abb.~\ref{fig:elektrodenanordnung_widerstand.PNG}).
        Die Messzelle wird dabei f\"ur beide Arten der Widerstandsmessung herangezogen,
        sie wird lediglich anders beschaltet.

        \bild{elektrodenanordnung_widerstand.PNG}{!h}{1}{Elektrodenanordnung der Messzelle}
        \section{Durchgangswiderstand}
            \subsection{\"Ubungsaufbau}
            Als Messger\"at wird ein Ohmmeter MILLI-TO 2 der Frima Dr. Kamphause 
            verwendet. Der Widerstand wird darauf digital angezeigt 
            (Messbereich: 50m$\Omega$ - 200 T$\Omega$; Messgenauigkeit: $\pm$1,5\%,
            $\pm$1 Digit bei 23 $^{\circ}$C). Abbildung 
            Abb.~\ref{fig:elektrodenanordnung_widerstand.PNG} zeigt die Messzelle und deren Aufbau.
            \\
            \\
            \bild{messaufbau_durchgang.PNG}{t}{1}{Beschaltung der Messzelle zum Messen des Durchgangswiderstandes}
            Abbildung Abb.~\ref{fig:messaufbau_durchgang.PNG} zeigt die Schaltung zum 
            Aufnehmen des Durchgangswiderstandes. Wichtig dabei ist anzumerken, dass auch 
            der Schutzring angeschlossen ist. Dies ist notwendig, da das Dielektrikum nicht nur 
            einen Durchgangswiderstand sonder auch einen Oberfl\"achenwiderstand. Dieser 
            erm\"oglicht auch eine Ladungstr\"agerbewegung \"uber die Oberfl\"ache der Probe.
            Um diesen parasit\"aren Oberfl\"achenstrom abzufangen, wird der Schutzring 
            mitbeschaltet. So wandern die Ionen nicht zur Gegenelektrode, sondern \"uber den Schutzring.
            \subsection{Ergebnisse und Erkenntnisse}
            \begin{table}[h]
                \begin{center}
                    \begin{tabular}{|l||l|l|}
                        \hline
                        Material (Dicke) & $R_D$ in $\Omega$ & $\rho_D$ in $\Omega$m \\
                        \hline
                        \hline
                        Plexiglas (5.35 mm) & $4.14 \cdot 10^{11}$ & $1.107 \cdot 10^{12}$ \\
                        \hline
                        Phenolharz (2.9 mm) & $1.43 \cdot 10^{9}$ & $2.641 \cdot 10^{9}$ \\
                        \hline
                        PVC (5 mm) & $1.2 \cdot 10^{9}$ & $1.283 \cdot 10^{10}$ \\
                        \hline
                        Teflon (5.4 mm) & $2.96 \cdot 10^{13}$ & $2.93 \cdot 10^{13}$ \\
                        \hline
                        Epoxidharz (1.4 mm) & $6.7 \cdot 10^{9}$ & $4.69 \cdot 10^{10}$ \\
                        \hline
                    \end{tabular}
                    \caption{Ergebnisse der Messung und errechneter Durchgangswiderstand}
                    \label{tab:table1}
                \end{center}
            \end{table}
            Die Messwerte wurden jeweils nach einer Minute im Betrieb aufgenommen. In Tabelle
            Tab.~\ref{tab:table1} kann man den gemessenen Durchgangswiderstand sowie den errechneten spezifischen
            Durchgangswiderstand sehen. Die errechneten Werten liegen ungef\"ahr bei den Werten des Skriptums.
            Beim Messen muss man vor allem darauf Aufpassen, dass sich Leitungen mit 
            unterschiedlichen Potentialen nicht ber\"uhren. Durch solche fehlerhaften Verlegungen
            der Messleitungen kann es zu bemerkbaren Ver\"anderungen des gemessenen Widerstandes
            kommen. Des weiteren ist bei dieser Messung die Vorgeschiche des Material auch relevant f\"ur die
            Ergebnisse. Die Widerstandsmessung basiert auf Ionenbewegung im Material. Die Ionen Sammeln sich
            dann auf einer Seite bei einer Messung. Wird dann das Material umgedreht, k\"onnen diese Ionen das 
            Material duchrwandern. Wird das Material nicht umgedreht zwischen den Messungen, so bleiben die Ionen
            auf der Seite und der Stromfluss ist geringer und somit der Widerstand gr\"o\ss{}er.
        \section{Oberfl\"achenwiderstand}
            \subsection{\"Ubungsaufbau}
            Bei diesem Teil der \"Ubung wird das gleiche Messger\"at, wie bei der Messung des Durchgangswiderstandes.
            Der Messaufbau bleibt ident, jedoch wird diesmal anstatt der zweiten Elektrode
            der Schutzring angeschlossen. Den genauen Messaufbau kann man Abbildung Abb.~\ref{fig:messaufbau_oberflaeche.PNG} entnehmen.
            \bild{messaufbau_oberflaeche.PNG}{h}{1}{Beschaltung der Messzelle zum Messen des Oberfl\"achenwiderstandes}
            \subsection{Ergebnisse und Erkenntnisse}
            \begin{table}[h]
                \begin{center}
                    \begin{tabular}{|l||l|l|}
                        \hline
                        Material (Dicke) & $R_S$ in $\Omega$ & $\rho_S$ in $\Omega$ \\
                        \hline
                        \hline
                        Plexiglas (5.35 mm) & $4.16 \cdot 10^{9}$ & $8.039 \cdot 10^{11}$ \\
                        \hline
                        Phenolharz (2.9 mm) & $1.16 \cdot 10^{9}$ & $2.2339 \cdot 10^{11}$ \\
                        \hline
                        PVC (5 mm) & $7.01 \cdot 10^{7}$ & $1.35 \cdot 10^{10}$ \\
                        \hline
                        Teflon (5.4 mm) & $1.17 \cdot 10^{11}$ & $2.263 \cdot 10^{13}$ \\
                        \hline
                        Epoxidharz (1.4 mm) & $4.6 \cdot 10^{7}$ & $8.89 \cdot 10^{9}$ \\
                        \hline
                    \end{tabular}
                    \caption{Ergebnisse der Messung und errechneter Oberfl\"achenwiderstand}
                    \label{tab:table3}
                \end{center}
            \end{table}
            Aus Tabelle Tab.~\ref{tab:table3} kann man die gemessenen Werte des Oberfl\"achenwiderstandes
            sowie die errechneten spezifischen Oberfl\"achenwiderst\"ande. Die errechneten Werte
            zeigen teilweise sehr starke Abweichungen gegen\"uber den Werten des Laborskriptums.
            Genau wie bei der Messung des Durchgangswiderstandes kommt es auch hier darauf
            an, dass sich die Messleitungen nicht ber\"uhren.
    \chapter{Dieelektrizit\"atszahl}
    \section{Grundlagen}
    Das makroskopische Verhalten von Dielektrika wird durch die Gleichung beschrieben.
    \begin{equation}
    \vv{D}=\varepsilon\vv{E}=\varepsilon_{0}\varepsilon_{r}\vv{E}
    \end{equation}
    $\vv{D}$ elektrische Flussdichte $[D]=As/m^{2}$ \\
    $\vv{E}$ elektrische Feldst\"arke $[E]=V/m$ \\
    $\varepsilon$ Permittivit\"at $[\varepsilon]=As/Vm$ \\
    $\varepsilon_{0}$ Permittivit\"at des Vakuums $[\varepsilon_{0}]=8,854\cdot10^{-12} As/Vm$ \\
    $\varepsilon_{r}$ relative Permittivit\"at $[\varepsilon_{r}]=1$ \\
    
    Die relative Permittivit\"at ist ein materialabh\"angiger Faktor. 
    Dieser gibt an um wie viel sich die elektrische Flussdichte \"andert 
    wenn man leeren Raum durch polarisierte Materie ersetzt. F\"ur den leeren 
    Raum ist dieser Faktor 1.
    \\
    Polarisation entsteht grunds\"atzlich durch das elektrische Feld und 
    der daraus resultierenden Krafteinwirkung zu Zustands\"anderungen, der 
    atomaren und molekularen Dipole. In einem Dielektriker k\"onnen sich die 
    Dipole nur so weit aus ihrer Gleichgewichtslage gebracht werden, bis die 
    R\"ucktreibende Kraft gleich jener ist wie jene, durch das Elektrische 
    Feld erzeugte. Da negativ und positiv geladene Teilchen sich 
    entgegengesetzt verschieben, wird jedes Volumelemnt zum elektrischen Dipol. 
    Das wird auch elektrische Polarisation, durch das elektrische Feld genannt. 
    Auf der gegen\"uberliegenden Seite, zur Polarisationsrichtung senkrechten 
    Oberfl\"ache, entsteht elektrische Ladung, jene Ladung nennt man scheinbare 
    Ladung, da sie sich nicht durch einen Leiter abf\"uhren l\"asst.
    Man unterscheidet zwischen 3 verschiedenen Polarisationsarten:\\
    \textbf{Elektronenpolarisation}\\
    Hierbei ergeben sich die Dipolelemente aus der Verschiebung der 
    negativ geladenen Elektronenh\"ullen gegen\"uber dem positiven geladenen 
    Atomkern. Die Einstellzeit betr\"agt dabei etwa $10^{-14}$ bis $10^{-15}$s 
    und kann daher auch noch sehr schnellen Feld\"anderungen folgen.\\
    \textbf{Atompolarisation}\\
    Diese Art tritt bei Molek\"ulen mit zumindest \"uberweigend 
    Ionenbindung auf. Durch die unterschieldichen Ladungen kommt es zu 
    ungleichen Verr\"uckungen der Ionen, wodurch Dipolelemente enstehen. Die 
    Einstellzeit betr\"agt $10^{-12}$ bis $12^{-13}$s.\\
    \textbf{Orientierungspolarisation}\\
    Dies tritt bei Stoffen auf, deren Molek\"ule bereits permanente elektrische Dipolmomente besitzen. Die resultierende Wirkung dieser Momente verschwindet jedoch ohne \"au\ss{}erem Feld. Grund daf\"ur ist, dass diese durch die W\"armebewegung statistisch verteilt werden.
    Erst durch ein elektrisches Feld kommt es zu einer Ausrichtung und es entsteht ein Gleichgewicht zwischen der thermisch bedingten Unordnung und dem durch das Feld verursachte Ordnungszustand. 
    Diese Art weist daher eine Temperaturabh\"angigkeit auf. Die Einstellzeit 
    liegt bei $10^{-9}$ bis $10^{-11}$s
    \bild{Polarisationsarten.png}{h}{0.7}{Arten der Polarisation}
    \pagebreak
    \section{Kapazit\"at eines Luftkondensators}
    \subsection{\"Ubungsdurchf\"uhrung}
    Es soll die Kapazit\"at des leeren Stoffmesskondensators (Luft als Dielektrikum) gemessen werden. Dabei wird der Plattenabstand von 1 bis 10mm in 1mm-Schritten vergr\"o\ss{}ert. Die gemessenen Werte sollen mit den berechneten Werten verglichen werden.
    Die Messung erfolgt mit einem QuadTEch 7600 LCR-Meter, an dem ein Stoffmesskondensator, Plattendurchmesse 100 mm, angeschlossen ist.
    \subsection{Ergebnisse}
    Wie in Abbildung Abb.~\ref{fig:Kapazitaet_vom_Plattenkondensator.png} erkennbar ist weichen die gemessenen Werte leicht von der errechneten ab. Diese Abweichung ist darauf zur\"uckzuf\"uhren, dass f\"ur die Berechnung von einem homogenen Feld ausgegangen wird. Jedoch bei der Messung St\"orungen im Randbereich auftreten und mitgemessen werden.
    \bild{Kapazitaet_vom_Plattenkondensator.png}{h}{1}{Kapazit\"at in Abh\"angigkeit der Luftspaltl\"ange}
    \section{Bestimmen der relativen Dielektrizit\"atszahl}
    \subsection{\"Ubungsdurchf\"uhrung}
    Es sollen zwei verschiedne Dielektrika-Proben mit unterschiedlicher Dicke ausgew\"ahlt werden und deren Kapazit\"at gemessen werden. Dann soll die scheinbare Dielektrizit\"atzahl, korrigierte Dielektritzit\"atzahl und die Luftspaltdicke berechnet werden.
    Daf\"ur wird wiederum der Stoffmesskondesator verwendet und mit dem QuadTech 7600 RLC-Meter gemessen.
    \bild{Luftspalt.png}{h}{0.5}{Schaltung zur Bestimmung des Luftspalts und der Dielektrizit\"atszahl}
    \subsection{Ergebnisse}
    Die Tabelle zeigt die gemessenen Kapazit\"aten der verschiedenen Proben und die daraus errechnete scheinbare Dielektrizit\"at.
    \\
	\begin{table}[h]
		\begin{center}
			\begin{tabular}{|c||c|c|c|}
				\hline
				Probe & Dicke in mm & Kapazit\"at in pF & scheinbare Dielektrizit\"atszahl\\
				\hline
				\hline
				Teflon & 1,1 & 134,87 & 2,2\\
				\hline
				Teflon & 3,42 & 455,46 & 2,26\\
				\hline
				Phenolharz & 0,54 & 857,65 & 6,97\\
				\hline
				Phenolharz & 1,46 & 295,48 & 8,45\\
				\hline
			\end{tabular}
                    \caption{Ergebnisse der Messung und errechnete scheinbare Dielektrizit\"atszahl}
                    \label{tab:table4}
		\end{center}
	\end{table}
    \\
    Die folgende Tabelle zeigt eine Gegen\"uberstellung der korrigierten Dielektrizit\"atszahl mit den Werten aus der Tabelle in der Aufgabenstellung, sowie die berechnete Luftspaltdicke.
    \\
	\begin{table}[h]
		\begin{center}
			\begin{tabular}{|c||c|c|c|}
				\hline
				Probe & Luftspaltdicke in $\mu$m & $\varepsilon_{r}$ errechnet& $\varepsilon_{r}$ Tabelle\\
				\hline
				\hline
				Teflon & 36,1 & 2,29 & 2,1\\
				\hline
				Phenolharz & 25,5 & 9,71 & 4,6 - 5,5\\
				\hline
			\end{tabular}
                    \caption{Errechnete Luftspalrbreite und Dielektrizit\"atszahl}
                    \label{tab:table5}
		\end{center}
	\end{table}
    \\
    F\"ur Teflon liegt der berechnete Wert 9\% \"uber dem Wert in der Tabelle. Diese geringe Abweichung ist auf Messungenauigkeiten und Unreinheiten der Proben zur\"uckzuf\"uhren.
    Beim Phenolharz ist der errechnete Wert fast doppelt so hoch wie der in der Tabelle. 
    
    \section{Frequenzabh\"angikeit von Kondesatoren}
    \subsection{\"Ubungsdurchf\"uhrung}
    Die Aufgabe besteht darin, dass man die Frequenzabh\"angigkeit der Kapaizit\"at verschiedener Kondensatoren misst.
    F\"ur die Messung wird das QuadTech 7600 LCR-Meter verwendet. Der 
    Frequenzbereich in dem die Kapazit\"aten gemessen werden sollen, 
    ist von 10 Hz bis 2 Mhz. Dabei muss besonderes Augenmerk auf die 
    Begrenzung des Messger\"ates gelegt werden. Der Messstrom des verwendeten 
    LCR-Meters kann in einem Bereich von 250 $\mu$A und 100 mA eingestellt 
    werden. Die Messspannung kann von 20mV bis 5V variiert werden. Ob die 
    Messgr\"o\ss{}en auch tats\"achlich in dem angegebenen Bereich liegen und die 
    Bedingungen der Messbr\"ucke eingehalten werden k\"onnen, h\"angt von der
    eingestellten Messfrequenz und von der angeschlossenen Messlast ab.
    
    \subsection{Ergebnisse}
    \subsubsection{Kondensator 1: 3900$\mu$F }
    In Abbildung Abb.~\ref{fig:Kondensatoren_muF.png} ist zu sehen, dass die Kapazit\"at mit steigender Frequenz sinkt. Im 
    niederfrequenten Bereich verl\"auft dies nur langsam, doch ab 500Hz f\"allt sie stark ab und geht 
    sogar ins negative. Diese Werte sind jedoch zu vernachl\"assigen, da sich das Messger\"at nicht mehr 
    im Arbeitsbereich befindet. Durch \"uberschlagsm\"a\ss{}ige Berechnung wurde ermittelt, dass der Messereich ab 
    ca. 12kHz verlassen wird.
    \subsubsection{Kondensator 2: 470$\mu$F }
    Die Kennlinie des zweiten Kondesators sieht \"ahnlich aus wie die des Kondesator 1. Hier setzt der Abfall 
    erst bei 1kHz ein. Der Arbeitsbereich des Messger\"ats wird bei ca. 7kHz verlassen
    \subsubsection{Kondensator 3: 4,7nF }
    Dieser Kondesator zeigt ein frequenzstabiles Verhalten. Der unterschied zwischen Maximum und Minimum liebt 
    bei 0,28nF. Seine Messergebnisse sind Abbildung Abb.~\ref{fig:Kondensatoren_nF.png} zu entnehmen.
    \subsubsection{Kondensator 4: 1$\mu$F }
    Auch dieser Kondensator hat \"uber einen gro\ss{}en Frequenzbereich einen stabilen Wert. Erst bei 700kHz weicht 
    der Wert stark ab und das Messger\"at verl\"asst den Arbeitsbereich.
    \bild{Kondensatoren_muF.png}{h}{0.85}{Kapazit\"at in Abh\"angigkeit der Frequenz ($\mu$F)}
    \bild{Kondensatoren_nF.png}{h}{0.9}{Kapazit\"at in Abh\"angigkeit der Frequenz (nF)}
    \chapter{Durchschlagsfestigkeit}
    \section{Grundlagen}
		Die Durchschlagsfestigkeit gibt die elektrische Feldst\"arke an, bei der in einem Isolierwerkstoff ein elektrischer Durchschlag 	
		erfolgt.
		Die Durchschlagsfestigkeit ist in vielen Anwendungen, wie etwa in der Hochspannungstechnik von enormer Bedeutung, da von ihr 		
		wesentliche Parameter wie zum Beispiel Leiterabst\"ande abh\"angen.
		\\
		Grunds\"atzlich unterscheidet man zwischen vier verschiedenen elektrischen Durchschlagsarten:
		\\
		\\
		\textbf{Lawinendurchschlag}
			\\
			Auch in nichtmetallischen Werkstoffen existieren wie in metallischen 
			Werkstoffen, alledings in viel geringer Anzahl,  vereinzelt freie Elektronen.
			Durch das Anlegen einer Spannung werden diese durch die auf sie einwirkende Feldst\"arke 
			beschleunigt und treffen eventuell auf ein anderes Atom.
			Ist die kinetische Energie des freien Elektrons gro\ss{} genug,
			entstehen bei einer Kollision neue freie Elektronen.
			Da sich dieser Prozess lawinenf\"ormig fortsetzt, spricht man hier vom Lawinendurchschlag.
			\\
			\\
		\textbf{Thermischer Durchschlag}
			\\
			Durch eine lokale Erhitzung im dielektrischen Werkstoff und der daraus resultierenden erh\"ohten Bewegung der Atome wird der
			elektrische Durchschlag ausgel\"ost, indem Ladungstr\"ager 
			freigesetzt werden.
			\\
			\\
		\textbf{Elektrolytischer Durchschlag}
			\\
			Bereits bei sehr kleinen Leckstr\"omen k\"onnen sich Metallionen baumartig an den Elektronen ablagern (Dendriten) und auf diese Weise 
			Stromleitpfade bilden. Daraus ergibt sich eine geringere Durchschlagsfestigkeit.
			\\
			\\
		\textbf{Gasentladungsdurchschlag}
			\\
			Grunds\"atzlich ist die Durchschlagsfestigkeit in Gasen geringer und in Gaseinschl\"ussen im Dielektrikum wirkt eine h\"ohere 
			elektrische Feldst\"arke. Deshalb kann es in Gaszwischenschichten zu Teilentladungen im Isolator kommen, welche den Isolierstoff sch
			\"adigen und in Folge dessen einen Durchschlag ausl\"osen.
		
	\section{Durchschlagsfestigkeit von Luft}

		\subsection{Aufgabenstellung}
			Bei dieser Aufgabe soll die Durchschlagsfestigkeit von Luft in 
            Abh\"angigkeit von den Elektrodenabst\"anden bestimmt werden. 
			Dazu werden zehn verschiedene Abst\"ande gew\"ahlt, bei denen jeweils 
			die Durchschlagsspannung gemessen wird. Die Durchschlagsfestigkeit ergibt sich infolgedessen aus dem Quotienten der Durchbruchspannung und der Luftspaltl\"ange.		\\
		\subsection{Messaufbau}
			Bei der Messung der Durchschlagsfestigkeit von Luft wird das 
			Hochspannungspr\"ufger\"at PGO-S3 von BAUR verwendet. 
			Alle notwendigen Grundfunktionen zur Spannungmessung werden vom
			Ger\"t selbstst\"andig ausgef\"uhrt. \\
			Die parallel angeordneten Elektroden des Ger\"ats weisen ein 
			Rogowski-Profil auf, die 
			einen m\"oglichst homogenen Verlauf der Feldst\"arke gew
			\"ahrleisten. 
			Die Pr\"ufung erfolgt mit einer sich stets erh\"ohenen, sinusf\"ormigen Wechselspannung. 
			Bei dieser Messung ist die Spannungs\"anderungsrate auf 0,5 kV/s eingestellt.
		\subsection{Messergebnisse}
		\bild{Durchschlagsspannung.png}{h}{0.8}{Abh\"angigkeit der Durchschlagsspannung von der Luftspaltl\"ange}
			
		Abbildung Abb.~\ref{fig:Durchschlagsspannung.png} zeigt den Zusammenhang zwischen den gemessenen Werten der Durchschlagsspannung und den Abstand. 
		Der Quotient dieser beiden Werte ergibt die elektrische Durchschlagsfeldst\"arke.
		\\
		Wie sich anhand der Grafik erkennen l\"asst, ist die Durchschlagsfestigkeit in kV/mm bei gr\"o\ss{}eren und kleineren Abst\"anden nicht gleich. 
		Dies liegt vor allem daran, dass  aufgrund der hohen Spannung bei gr\"o\ss{}eren Abst\"anden an den Elektroden ein st\"arkeres elektrisches Feld entsteht. 
		Dies erleichert wiederum die Ionisation und f\"uhrt somit zum verfr\"uhten Durchschlag bei hohen Spannungen.
	\section{Durchschlagsfestigkeit von Papier}
	
		\subsection{Aufgabenstellung}
		Hier wird die Durchschlagsfestigkeit von Papier bei einer Spannungs\"anderungsrate von 0,5 kV/s ermittelt. 
		Die Diche des Papierstapels wird von 2mm bis 5mm in 1mm-Schritten gesteigert.
		\subsection{Messaufbau}
		Zur Messung wird wie bei der Messung der Durchschlagsfestigkeit von Luft das PGO-S3 von BAUR verwendet. 
		Bei dieser Messung wird allerdings der Papierstapel zwischen den Elektroden eingespannt.
		Dabei ist es wichtig, das Papier m\"oglichst ohne Lufteinschl\"usse einzuspannen, da dieser ansonsten zu einem Gasentladungsdurchschlag f\"uhren kann, was wiederum zu einer Verf\"alschung des Messergebnisses f\"uhrt.
		\subsection{Messergebnisse}
		\begin{table}[h]
			\begin{center}
				\begin{tabular}{|c||c|c|}
					\hline
					Abstand in mm & Durchschlagsspannung in kV & Durschlagsfeldst\"arke in kV/mm \\
					\hline
					\hline
					2mm & 14,8 kV & 7,4 kV/mm \\
					\hline
					3mm & 16,9 kV & 5,6 kV/mm \\
					\hline
					4mm & 18,8 kV & 4,7 kV/mm \\
					\hline
					5mm & 22,9 kV & 4,58 kV/mm \\
					\hline				
				\end{tabular}
			\caption{Durchschlagsfeldst\"arke in Abh\"angigkeit von der Paierdicke}
			\label{tab:table2}
			\end{center}
		\end{table}
		Durch Auswertung der Tabelle Tab.~\ref{tab:table2} erh\"alt man eine Durchschlagsfestgkeit des Papiers von 7,4 kV/mm bei d\"unnen Papierschichten und eine Durchschlagsfestigkeit von 4,58 kV/mm bei dickeren Papierschichten.
		Hier l\"asst sich wie bei der Messung der Durchschlagsfestigkeit von Luft eine umgekehrte Proportionalit\"at feststellen. 
		
		
	\section{Messung der 1-Minuten Stehspannung von Papier}
	
		\subsection{Aufgabenstellung}
		Zur Durchf\"uhrung dieser Messung werden 40 \% der bereits ermittelten Durchschlagsspannung zur verwendeten Papierdicke eingestellt. 
		Die an den Elektroden anliegende Spannung wird anschlie\ss{}end im Minutentakt um jeweils 8\% erh\"oht. 
		Der letzte Spannungswert vor dem elektrischen Durchschlag wird 1-Minuten Stehspannung genannt. 
		\subsection{Messaufbau}
		Wie bei den vorhergehenden Messungen wird das PGO-S3 Messger\"at von BAUR verwendet. 
		Der Wert der Startspannung, also den 40\% der Durchschlagsspannung, betr\"agt bei einer Papierdicke von 3mm 6,8 kV.
		Anschlie\ss{}end wird die an den Elektroden des Messger\"ates anliegende Spannung nach jeder Minute um 8\% gesteigert. 
		\subsection{Messergebnisse}
		Der Durchbruch erfolgt bei dieser Messung bereits bei 12,2 kV, also 4,7 kV vor der eigentlichen Durchbruchspannung.
		Dies liegt vor allem daran, dass der Isolator, also in diesem Fall der 3mm dicke Papierstapel, bereits einige Zeit mit einer hohen Spannung belastet worden ist.
		Deshalb erfolgte der \"Uberschlag bereits fr\"uher. Die 1-Minuten Stehspannung betr\"agt also 12,2 kV.
\listoffigures	
\end{document}